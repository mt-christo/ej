%%%%%%%%%%%%%%%%%%%%%%%%%%%%%%%%%%%%%%%%%
% baposter Landscape Poster
% LaTeX Template
% Version 1.0 (11/06/13)
%
% baposter Class Created by:
% Brian Amberg (baposter@brian-amberg.de)
%
% This template has been downloaded from:
% http://www.LaTeXTemplates.com
%
% License:
% CC BY-NC-SA 3.0 (http://creativecommons.org/licenses/by-nc-sa/3.0/)
%
%%%%%%%%%%%%%%%%%%%%%%%%%%%%%%%%%%%%%%%%%

%----------------------------------------------------------------------------------------
%	PACKAGES AND OTHER DOCUMENT CONFIGURATIONS
%----------------------------------------------------------------------------------------

\documentclass[portrait, a4paper, fontscale=1.5, margin=2em]{baposter} % Adjust the font scale/size here

\usepackage{array}
\newcolumntype{L}{>{\centering\arraybackslash}m{6cm}}
\newcolumntype{M}{>{\centering\arraybackslash}m{12cm}}
\usepackage{graphicx} % Required for including images
\graphicspath{{figures/}} % Directory in which figures are stored

\usepackage{amsmath} % For typesetting math
\usepackage{amssymb} % Adds new symbols to be used in math mode

\usepackage{booktabs} % Top and bottom rules for tables
\usepackage{enumitem} % Used to reduce itemize/enumerate spacing
\usepackage{palatino} % Use the Palatino font
\usepackage[font=small,labelfont=bf]{caption} % Required for specifying captions to tables and figures

\usepackage{multicol}% Required for multiple columns
\setlength{\columnsep}{1em} % Slightly increase the space between columns
\setlength{\columnseprule}{0mm} % No horizontal rule between columns

\usepackage{tikz} % Required for flow chart
\usetikzlibrary{shapes,arrows} % Tikz libraries required for the flow chart in the template

\newcommand{\compresslist}{ % Define a command to reduce spacing within itemize/enumerate environments, this is used right after \begin{itemize} or \begin{enumerate}
\setlength{\itemsep}{1pt}
\setlength{\parskip}{0pt}
\setlength{\parsep}{0pt}
}

\begin{document}

\begin{poster}
{
columns=1,
headerborder=none, % Adds a border around the header of content boxes
colspacing=3em, % Column spacing
bgColorOne=white, % Background color for the gradient on the left side of the poster
bgColorTwo=white, % Background color for the gradient on the right side of the poster
borderColor=white, % Border color
headerColorOne=white, % Background color for the header in the content boxes (left side)
headerColorTwo=white, % Background color for the header in the content boxes (right side)
headerFontColor=white, % Text color for the header text in the content boxes
boxColorOne=white, % Background color of the content boxes
textborder=none, % Format of the border around content boxes, can be: none, bars, coils, triangles, rectangle, rounded, roundedsmall, roundedright or faded
eyecatcher=true, % Set to false for ignoring the left logo in the title and move the title left
headerheight=0.1\textheight, % Height of the header
headershape=roundedright, % Specify the rounded corner in the content box headers, can be: rectangle, small-rounded, roundedright, roundedleft or rounded
headerfont=\Large\bf\textsc, % Large, bold and sans serif font in the headers of content boxes
textfont=\Large,
%textfont={\setlength{\parindent}{1.5em}}, % Uncomment for paragraph indentation
linewidth=0pt % Width of the border lines around content boxes
}
%----------------------------------------------------------------------------------------
%	TITLE SECTION 
%----------------------------------------------------------------------------------------
%
{\includegraphics[height=2.5em]{logo_novo_x.png}} % First university/lab logo on the left
{\bf\textsc{Segment performance}\vspace{0.6em}} % Poster title
{\textsc{ Market Cap rated \hspace{0.3em}-\hspace{0.3em} Top \Sexpr{top_mcap} companies}} % Author names and institution
{\includegraphics[height=2.5em]{logo_novo_x.png}} % Second university/lab logo on the right

\headerbox{Introduction}{name=a1, column=0}{
\begin{center}
\includegraphics{novo_chart.png}
\end{center}
}

\headerbox{Objectives}{name=a2, column=0, below=a1}{
\begin{center}
Segment performance metrics:\\[1.3em]
\begin{tabular}{L | c | c | c | M}
\toprule
Segment & 1Y Sigma & 1Y Return & Total Return & Current Basket\\
\midrule
\Sexpr{my_table}
\bottomrule
\end{tabular}
\end{center}
}


%----------------------------------------------------------------------------------------

\end{poster}

\end{document}
